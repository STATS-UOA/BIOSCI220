\documentclass{article}
\usepackage{graphicx}

\begin{document}

\title{Learning Objectives weeks 7--11, 27th April--29th May}
\author{BIOSCI220}
\date{}
\maketitle


\section*{What we assume the students are able to do}
\begin{itemize}
\item Identify the following types of variables in the dataset
  \begin{itemize}
  \item factor
  \item continuous
  \item categorical
  \end{itemize}
\item Write their own R code to summarise each variable using an appropriate plot. Specifically be able to produce the following plots
  \begin{itemize}
  \item boxplot
  \item scatter plot
  \item histogram
  \item barplot
  \end{itemize}
\item Represent their data as a clean data frame in R and manipulate it accordingly (using tidyverse??)
\item Communicate a summary of the dataset accurately and concisely
\item Communicate any limitations relevant to the dataset
\end{itemize}


\section*{Week 7 (beginning 27th April):  Experimental design}
By the end of this lab student should be able to
\begin{itemize}
\item Describe the setup of the following experimental designs
  \begin{itemize}
  \item independent measures
  \item repeated measures
    \item matched pairs
  \end{itemize}
\item Identify the type of design used in the given case studies
  \item Discuss the advantages and disadvantages of different design types
\item Critique experimental designs used in the the given case studies
\item Design an experiment that follows good experimental design principles (interactive experiment in lab?)
\end{itemize}

\section*{Week 8 (beginning 4th May): Visualising and analysing multivariate data}
By the end of this lab student should be able to
\begin{itemize}
\item Discuss the aims and motivations of Multidimensional Scaling (MDS) and its relevance in biology
\item List and summarise the three main types of MDS:
  \begin{itemize}
  \item classical MDS
  \item metric MDS
  \item non-metric MDS
  \end{itemize}
\item Write R code to create an MDS plot appropriate to the given dataset
\item Interpret an MDS plot
\item Explain the aims and motivation behind Principal Component Analysis (PCA) and its relevance in biology 
\item Write R code to carry out PCA
  \item Interpret the effectively communicate the output of a PCA 
\end{itemize}



\section*{Week 9 (beginning 11th May): Hypothesis testing}
By the end of this lab student should be able to
\begin{itemize}
\item List appropriate questions posed by the biological questions and  outline an appropriate hypothesis test that would answer it
\item Describe the aims of the following hypothesis tests
  \begin{itemize}
  \item one-sample t-test
  \item two-sample t-test
  \item randomization test
  \end{itemize}
\item List the aims of hypothesis testing and write out the appropriate null and alternative hypothesis using statistical notation
\item Write R code to carry out an hypothesis test using the appropriate variables in their dataset. Specifically write R code to carry out
  \begin{itemize}
  \item one-sample t-test
  \item two-sample t-test
  \item randomization test
  \end{itemize}
\item Interpret and communicate the findings of an hypothesis test accurately and concisely
\item List the limitations of the hypothesis in relation to the questions posed by the data
\end{itemize}


\section*{Week 10 (beginning 18th May): Introduction to linear modelling}
By the end of this lab student should be able to
\begin{itemize}
\item Develop a biologically relevant question of interest  from the dataset and identify the following types of variables in the dataset
  \begin{itemize}
  \item response variable
  \item explanatory variables
  \end{itemize}
\item Express their question of interest accurately and concisely
\item Carry out an interpret tests for the existence of relationships between explanatory variables and the response in a linear model
\item Write R code to fit a linear model with a single continuous explanatory variable
\item Write R code to fit a linear model with a continuous explanatory variable and a factor explanatory variable
\item Interpret estimated effects with reference to confidence intervals from linear regression models. Specifically the interpretation of
  \begin{itemize}
  \item the intercept
  \item the effect of a factor
  \item the effect of a one-unit increase in a numeric variable
  \item the effect of an x-unit increase in a numeric variable
  \end{itemize}
\item Critique the fitted model
\end{itemize}

\section*{Week 11 (beginning 25th May): Modelling II}
By the end of this lab student should be able to
\begin{itemize}
\item Write R code to fit a linear model with interaction terms in the explanatory variables
\item Interpret estimated effects with reference to confidence intervals from linear regression models. Specifically the interpretation of
  \begin{itemize}
  \item main effects in a model with an interaction
  \item the effect of one variable when others are included in the model
  \end{itemize}
\item Explain why you may want to include interaction effects in a linear model
\item Describe the differences between the operators : and * in R model-fitting formulae
\item Critique the fitted model
\end{itemize}



\end{document}
