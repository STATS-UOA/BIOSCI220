\documentclass{article}
\usepackage{graphicx}
\usepackage{hyperref}
\hypersetup{
    colorlinks=true,
    linkcolor=blue,
    filecolor=magenta,      
    urlcolor=cyan,
}
\usepackage[margin=0.5in]{geometry}
\begin{document}

\title{Course outline weeks 7--11\\ 4th May--5th June}
\author{BIOSCI220}
\date{}
\maketitle


\section*{Students are expected to be able to}
\begin{itemize}
\item Identify the following types of variables in the dataset
  \begin{itemize}
  \item discrete
  \item continuous
  \end{itemize}
\item Write their own R code to summarise each variable using an appropriate plot. Specifically be able to produce the following plots
  \begin{itemize}
  \item boxplot
  \item scatter plot
  \item histogram
  \item barplot
  \end{itemize}
\item Represent their data as a clean data frame in R and manipulate it accordingly. Use of [ \& : etc.
\item Communicate a summary of the dataset accurately and concisely
\item Communicate any limitations relevant to the dataset
\end{itemize}

\begin{center}
   \href{https://b.socrative.com/login/student/}{\textbf{\Large Short pre-lecture quiz}} (Room Name: BIOSCI220, not assessed)
  
\end{center}


\newpage


\section*{Week 7 (beginning 4th May):  Experimental design}
\subsection*{Learning objectives}
By the end of this lab student should be able to
\begin{itemize}
\item Identify the following experimental variables
  \begin{itemize}
      \item independent variable
      \item dependent variable
        \end{itemize}
\item List and describe Fisher's three principals of experimental design
  \begin{itemize}
       \item Randomization
       \item Replication
       \item Blocking
       \end{itemize}  
   \item Identify the design concepts used in given case studies
   \item Discuss the advantages and disadvantages of different designs
   \item Critique experimental designs
\end{itemize}

\subsection*{Mini lectures}

\begin{itemize}
\item Randomization
\item Replication
\item Local Control (blocking etc.)
\item Lab tutorial explained (design your own experiment, peer review)
\end{itemize}

\subsubsection*{Assessed quiz on CANVAS worth 1\% of your final grade}

\begin{center}
  \href{https://b.socrative.com/login/student/}{\textbf{\Large Practise quiz}} (Room Name: BIOSCI220, not assessed, available the week before)
  
\end{center}

\subsection*{Lab---worth 6\% of your final grade}

During this exercise you will design your own experiment (you are not expected to carry this out!). Filling out the CANVAS worksheet you should 

\begin{itemize}
\item concisely summarize what the question is you have you wish to answer during your experiment,

\item summarize your experimental design, paying particular attention to Fisher's 3 principles of experimental design.
\end{itemize}

You will also be expected to peer review 4 other worksheets; this also means that your work will be peer marked by 4 of your classmates. The worksheets you will review will be automatically assigned to you after the assignment due date. When carrying out this peer review please follow the rubric carefully and be mindful that your comments, although anonymous, will be passed on to your peer.


\newpage

\subsection*{Week 8 (beginning 11th May): Visualising and analysing multivariate data}
\subsection*{Learning objectives}
By the end of this lab student should be able to
\begin{itemize}
\item Discuss the aims and motivations of Multidimensional Scaling (MDS) and its relevance in biology
   \item Explain the aims and motivation behind Principal Component Analysis (PCA) and its relevance in biology
   \item Write R code to carry out PCA
   \item Interpret the effectively communicate the output of PCA
\end{itemize}

\subsection*{Mini lectures}

\begin{itemize}
\item Intro to MDS and scaling
\item When the distances are Euclidean (PCA \& bread)
\item Carrying out and drawing inference PCA in `R`
\item Lab tutorial explained (PCA cheatsheet, peer review)
\end{itemize}

\subsubsection*{Assessed quiz on CANVAS worth 1\% of your final grade}

\begin{center}
  \href{https://b.socrative.com/login/student/}{\textbf{\Large Practise quiz}} (Room Name: BIOSCI220, not assessed, available the week before)
  
\end{center}



\subsection*{Lab---worth 6\% of your final grade}

Using a dataset of your choice you should make a cheatsheet illustrating the concepts of PCA and the steps required in the analysis. You may include R code and any plots you feel necessary. The text should be minimal, but easy to follow by your peers. 

\begin{itemize}

\item Your cheatsheet should briefly
  \begin{enumerate}
  \item Explain the aims of Principal Component Analysis
  \item Contain reproducible R code to carry out PCA
  \item Explain any inferences drawn
  \end{enumerate}
  
\item Be very concise; rely on diagrams where possible.

\item Pay attention to the details!

\item Code comments inform, but fail to draw the readers attention. It is better to use arrows, speech bubbles, etc. for important information. If it is not important information, leave it out.

\item Simple working examples are more helpful than documentation details.

\item Add some concise text to help the user make sense of your sections, diagrams and inferences.
\end{itemize}

You may use whatever software you want to create your cheatsheet, however, you must export the file as a PDF before uploading it to CANVAS. You will also be expected to peer review 4 other cheatsheets; this also means that your work will be peer marked by 4 of your classmates. The cheatsheets you will review will be automatically assigned to you after the assignment due date. When carrying out this peer review please follow the rubric carefully and be mindful that your comments, although anonymous, will be passed on to your peer.

\newpage

\section*{Labs 9--11 will be assessed via a one page Executive Summary.}

This Executive Summary will be worth  18\% of your final grade and is in addition to he weekly CANVAS quiz. During lab time you are free to work through the material provided and your final report. You may work in groups (digitally), however, the final report must be your \textbf{own} work. Any plagiarism will automatically result in 0\% for the report. Your Executive Summary should be no more than one A4 page. If should concisely effectively communicate your hypothesis, the statistical analysis undertaken, and your findings. Here are some guidelines to follow when writing your executive summary
\begin{itemize}
\item  It should \textbf{not} contain any statistical terminology that would only be properly understood by a statistician
  \item Recall there are a set of main messages that you should report from your analysis, the reader doesn't need to know about all the work you carried out
\item A brief outline of an Executive Summary should follow the sections listed below
  \begin{itemize}
\item Introduction: a one or two sentence description of the data and the purpose of the analysis
\item Methods: important \textbf{non technical} information for the reader about the analysis carried out
\item Report findings and the strength of evidence for them
\item Quantification: how reliable/generalisable are those findings
  \item Summary: a one or two sentence summary of the major findings
  \end{itemize}
\end{itemize}

\newpage

\subsection*{Week 9 (beginning 18th May): Hypothesis testing}


\subsection*{Learning objectives}
By the end of this lab student should be able to
\begin{itemize}
\item List appropriate questions posed by the biological questions and  outline an appropriate hypothesis test that would answer it
\item Describe the aims of the following hypothesis tests
  \begin{itemize}
  \item one-sample t-test
  \item two-sample t-test
  \item randomization test
  \end{itemize}
\item List the aims of hypothesis testing and write out the appropriate null and alternative hypothesis using statistical notation
\item Write R code to carry out an hypothesis test using the appropriate variables in their dataset. Specifically write R code to carry out
  \begin{itemize}
  \item one-sample t-test
  \item two-sample t-test
  \item randomization test
  \end{itemize}
\item Interpret and communicate the findings of an hypothesis test accurately and concisely
\item List the limitations of the hypothesis in relation to the questions posed by the data
\end{itemize}

\subsection*{Mini lectures}

\begin{itemize}
\item Hypotheses, why?
\item Differences in mean
\item Randomization tests
\item Lab tutorial explained (executive summary)
\end{itemize}

\subsubsection*{Assessed quiz on CANVAS worth 1\% of your final grade}

\begin{center}
  \href{https://b.socrative.com/login/student/}{\textbf{\Large Practise quiz}} (Room Name: BIOSCI220, not assessed, available the week before)
  
\end{center}




\newpage

\section*{Week 10 (beginning 25th May): Introduction to linear modelling}
\subsection*{Learning objectives}
By the end of this lab student should be able to
\begin{itemize}
\item Develop a biologically relevant question of interest  from the dataset and identify the following types of variables in the dataset
  \begin{itemize}
  \item response variable
  \item explanatory variables
  \end{itemize}
\item Express their question of interest accurately and concisely
\item Carry out and interpret tests for the existence of relationships between explanatory variables and the response in a linear model
\item Write R code to fit a linear model with a single continuous explanatory variable
\item Write R code to fit a linear model with a continuous explanatory variable and a factor explanatory variable
\item Interpret estimated effects with reference to confidence intervals from linear regression models. Specifically the interpretation of
  \begin{itemize}
  \item the intercept
  \item the effect of a factor
  \item the effect of a one-unit increase in a numeric variable
  \item the effect of an x-unit increase in a numeric variable
  \end{itemize}
\item Make a point prediction of the response for a new observation
\end{itemize}

\subsection*{Mini lectures}

\begin{itemize}
\item ANOVA $\equiv$ regression
\item Explanatory variables and the response
\item Fitting and interpreting linear models in R
\end{itemize}


\subsubsection*{Assessed quiz on CANVAS worth 1\% of your final grade}

\begin{center}
  \href{https://b.socrative.com/login/student/}{\textbf{\Large Practise quiz}} (Room Name: BIOSCI220, not assessed, available the week before)
  
\end{center}


\newpage

\subsection*{Week 11 (beginning 1st June): Modelling II}
\subsection*{Learning objectives}
By the end of this lab student should be able to
\begin{itemize}
\item Write R code to fit a linear model with interaction terms in the explanatory variables
\item Interpret estimated effects with reference to confidence intervals from linear regression models. Specifically the interpretation of
  \begin{itemize}
  \item main effects in a model with an interaction
  \item the effect of one variable when others are included in the model
  \end{itemize}
\item Explain why you may want to include interaction effects in a linear model
\item Describe the differences between the operators : and * in R model-fitting formulae
\item Critique the fitted model
\end{itemize}

\subsection*{Mini lectures}

\begin{itemize}
\item Multiple explanatory variables
\item Interactions
\item Model diagnostics
\end{itemize}

\subsubsection*{Assessed quiz on CANVAS worth 1\% of your final grade}
\begin{center}
  \href{https://b.socrative.com/login/student/}{\textbf{\Large Practise quiz}} (Room Name: BIOSCI220, not assessed, available the week before)
  
\end{center}

\end{document}
